\documentclass[a4paper, 12pt]{article}
\usepackage[utf8]{inputenc}
\usepackage[T2A]{fontenc}
\usepackage{amsmath,amssymb,amsthm}
\usepackage[a4paper,hmargin=2.5cm,vmargin=2.5cm]{geometry}

\newcommand{\Image}{\text{Im}\;}
\newcommand{\Ker}{\text{Ker}\;}

\begin{document}

\section{Theory}
\subsection{Persistent Homology and Giant Cycles}
\par The first we should understand, which objects topological data analysis research.
\par People call cloud a collection of points $\{x_\alpha\} \subset X$, where $X$ is a metric space. That's interesting to convert that data to some structure, so points can be representated as vertices of some combinatorial graph. And that graph can become scaffold of a simplicial complex. That's a good way to research data, ignoring high dimension of the space \cite{ghrist}.
\par A simplicial complex $X$ (I mean abstract simplicial complex) is a set of vertices $\{v_\alpha\}$ and a collection of its subsets, called simplices (simplex $a$ which is set of elements $k$ have dimension $k-1$ and can be called $k$-simplex), $X$ such that: for all $a\in X$ for all $b\subset a$ $b\in S$ \cite{prasolov}. The dimension of a simplicial complex is the maximal dimension of it's simplices ($\dim X = \max\limits_{a\in S} \dim a =\max\limits_{a\in S}|a| - 1$).
\par When simplicial complex is defined, let's continue way to define simplicial homology. Let $\Delta_n(X)$ be the free abelian group with basis on $n$-simplices $e_\alpha^n$ of $X$.  That groups elements can be rewritten as $\sum_\alpha n_\alpha e_\alpha^n$ and called $n$-chains. We also can represent them as $\sum_\alpha n_\alpha \sigma_\alpha$ where the $\sigma_\alpha: \Delta^n\to X$ is the characteristic map of .%hatcher-105
\par The boundary of $n$-simplex $(v_0, ..., v_n)$ is $(n-1)$-simplices $[v_0, ..., v_{i-1}, v_{i+1}, ..., v_n]$. So let the boundary be $\sum_i (-1)^i F_i$. The signs are inserted to take orientations into account, so that all the faces of a simplex are coherently oriented. Using that geometry we can define a boundary homomorphism $\delta: \Delta_n(X) \to \Delta_{n-1}(X)$:
$$
	\delta_n(\sigma_\alpha) = \sum_i (-1^i)\sigma_\alpha \;:\;
	[v_0, ..., v_{i-1}, v_{i+1}, ..., v_n]
$$
\par So there is the lemma, which said that the composition $\Delta_n(X) \xrightarrow{\delta_n} \Delta_{n-1}(X) \xrightarrow{\delta_{n-1}} \xrightarrow{\delta_{n-2}} \Delta_{n-2}(X)$ is zero. That's not hard to prove. We have 
$$\delta_n(\sigma) = \sum_i (-1)^i \sigma \;:\; [v_0, ..., v_{i-1}, v_{i+1}, ..., v_n]$$
and hence
$$
\begin{matrix}
	\delta_{n-1}\delta_{n} = 
	\sum\limits_{j<i} (-1)^i(-1)^{j} \sigma : [v_0, ..., v_{j-1}, v_{j+1}, ..., v_{i-1}, v_{i+1}, ..., v_n] \\
	+ 
	\sum\limits_{j>i} (-1)^i(-1)^{j-1} \sigma : [v_0, ..., v_{i-1}, v_{i+1}, ..., v_{j-1}, v_{j+1}, ..., v_n] = 0
\end{matrix}
$$
\par So we have a sequence of homomorphisms of abelian groups
$$
	\cdots \to 
	C_{n+1}  \xrightarrow{\delta_{n+1}}
	C_{n}  \to
	\cdots \to 
	C_{1}  \xrightarrow{\delta_{1}}
	C_{0}  \xrightarrow{\delta_{0}} 0
$$
such that $\delta_n\delta_{n+1} = 0$ for each $n$. Sequences like that are called chain complexes. Cause the equation $\delta_n\delta_{n+1} = 0$ is equivalent to the inclusion $\Image \delta_{n+1} \subset \Ker \delta_n$, we can defind the $n$-th homology group of the chain complex as the quotient group $H_n = \Ker\delta_n/\Image\delta_{n+1}$. In the case of simplicial complex $C_n = \Delta_n(X)$, so the homology group $\Ker\delta_n/\Image\delta_{n+1}$ be called the $n$-th homology group of $X$ and can be noted $H_n^\Delta(X)$. People call the elements of $\Ker\delta_n$ cycles and the elements of $\Image\delta_{n+1}$ boundaries. \cite{hatcher}

\par One of the natural methods to represent a cloud as a simplicial complex is the Cech complex.  For a given cloud $\{x_\alpha\}\subset\mathbb{E}^n$ the Cech complex $C_\epsilon$ is the simplicial complex whose $k$-simplices (the simplices dimension $k$: $a: |a|-1 = k$) are determined by unordered $(k+1)$-tuples of points $\{x_\alpha\}_0^k$ whose closed $\epsilon/2$-ball neighbourhoods have a point of common intersection \cite{ghrist}.
\par Another one natural method to represent a cloud as a simplicial complex is the Rips complex. For a given cloud $\{x_\alpha\}\subset\mathbb{E}^n$ the Rips complex $R_\epsilon$ is determined by unordered $(k+1)$-tuples of points whose for each pair of points $\{x_\alpha\}_0^k$ the distance between that pairs points less or equal $\epsilon$.
% picture
\par ... (the Cech (Nerve) theorem) 
\par ...




\subsection{Lattices Voronoi Cells and their Interpretations on Thorus}
\par In this chapter we will show default definitions about lattices in $\mathbb{R}^n$, talk about Voronoi cells and then extropolate their definitions to the thorus case. 
\par A lattice in $\mathbb{R}^n$ is a subset $\Gamma\subset\mathbb{R}^n$ with the property that there exxists a basis $(e_1, ..., e_n)$ of $\mathbb{R}^n$ s.t. $\Gamma = \mathbb{Z}e_1 \oplus\cdots\oplus \mathbb{Z}e_n$. \cite{ebeling}
\par Let's throw few examples of lattices, which will be interesting in this work:
\par The Lattices $Z_n$: ...
\par The Lattices $A_n$: ...
\par The Lattices $D_n$: ...

\par Let's define a $\Gamma^*$ dual to $\Gamma$ as $\{x\in\mathbb{R^n} :\; x\cdot y\in\mathbb{Z} \; \forall y\in\Gamma\}$.
\par ...

\par Let's define $d$-dimensional thorus as $\mathbb{R}^n/\mathbb{Z}^n$ or $\left(\mathbb{R}/\mathbb{Z}\right)^n$. Not hard to see, that $\mathbb{R}/a_1\mathbb{Z}\times\cdots\times\mathbb{R}/a_n\mathbb{Z}$ ($a_1,..., a_n\in\mathbb{R}_{>0}$) will be the homeomorphically-same object.
\par Let's redefine lattice thinking, that' lattices lie not just on $\mathbb{R}^n$, but on some thorus with defined equivalence relation. So...

\subsection{Random Filtration on Cells}
\par ...



\addcontentsline{toc}{section}{References}
\begin{thebibliography}{}
\bibitem{hatcher} Hatcher, A. (2001). Algebraic topology. Proceedings of The Edinburgh Mathematical Society - PROC EDINBURGH MATH SOC. 46. 511-512. 10.1017/S0013091503214620. 
\bibitem{ebeling} Ebeling, Wolfgang. (2002). Lattices and Codes. 10.1007/978-3-322-90014-2.
\bibitem{prasolov}  Prasolov, V. V. (2006), Elements of combinatorial and differential topology, American Mathematical Society, ISBN 0-8218-3809-1, MR 2233951
\bibitem{ghrist} Ghrist, Robert. (2008). Barcodes: The persistent topology of data. BULLETIN (New Series) OF THE AMERICAN MATHEMATICAL SOCIETY. 45. 10.1090/S0273-0979-07-01191-3. 
\end{thebibliography}




\end{document}